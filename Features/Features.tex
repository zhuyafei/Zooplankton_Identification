\documentclass[12pt]{article}
%%---------------------------------------------------------------------
% packages
% geometry
\usepackage{geometry}
% font
\usepackage{fontspec}
\defaultfontfeatures{Mapping=tex-text}  %%如果没有它,会有一些 tex 特殊字符无法正常使用,比如连字符。
\usepackage{xunicode,xltxtra}
\usepackage[BoldFont,SlantFont,CJKnumber,CJKchecksingle]{xeCJK}  % \CJKnumber{12345}: 一万二千三百四十五
\usepackage{CJKfntef}  %%实现对汉字加点、下划线等。
\usepackage{pifont}  % \ding{}
% math
\usepackage{amsmath,amsfonts,amssymb}
% color
\usepackage{color}
\usepackage{xcolor}
\definecolor{EYE}{RGB}{199,237,204}
\definecolor{FLY}{RGB}{128,0,128}
\definecolor{ZHY}{RGB}{139,0,255}
% graphics
\usepackage[americaninductors,europeanresistors]{circuitikz}
\usepackage{tikz}
\usetikzlibrary{positioning,arrows,shadows,shapes,calc,mindmap,trees,backgrounds}  % placements=positioning
\usepackage{graphicx}  % \includegraphics[]{}
\usepackage{subfigure}  %%图形或表格并排排列
% table
\usepackage{colortbl,dcolumn}  %% 彩色表格
\usepackage{multirow}
\usepackage{multicol}
\usepackage{booktabs}
\usepackage{tcolorbox}
% code
\usepackage{fancyvrb}
\usepackage{listings}
\lstset{language=C++}%这条命令可以让LaTeX排版时将C++键字突出显示
\lstset{breaklines}%这条命令可以让LaTeX自动将长的代码行换行排版
\lstset{extendedchars=false}
% title
\usepackage{titlesec}
% head/foot
\usepackage{fancyhdr}
% ref
\usepackage{hyperref}
% pagecolor
\usepackage[pagecolor={EYE}]{pagecolor}
% tightly-packed lists
\usepackage{mdwlist}

\usepackage{../styles/iplouccfg}
\usepackage{../styles/zhfontcfg}
\usepackage{../styles/iplouclistings}

%%---------------------------------------------------------------------
% settings
% geometry
\geometry{left=2cm,right=1cm,top=2cm,bottom=2cm}  %设置 上、左、下、右 页边距
\linespread{1.5} %行间距
% font
\setCJKmainfont{Adobe Kaiti Std}
%\setmainfont[BoldFont=Adobe Garamond Pro Bold]{Apple Garamond}  % 英文字体
%\setmainfont[BoldFont=Adobe Garamond Pro Bold,SmallCapsFont=Apple Garamond,SmallCapsFeatures={Scale=0.7}]{Apple Garamond}  %%苹果字体没有SmallCaps
\setCJKmonofont{Adobe Fangsong Std}
% graphics
\graphicspath{{../figures/}}
\tikzset{
    % Define standard arrow tip
    >=stealth',
    % Define style for boxes
    punkt/.style={
           rectangle,
           rounded corners,
           draw=black, very thick,
           text width=6.5em,
           minimum height=2em,
           text centered},
    % Define arrow style
    pil/.style={
           ->,
           thick,
           shorten <=2pt,
           shorten >=2pt,},
    % Define style for FlyZhyBall
    FlyZhyBall/.style={
      circle,
      minimum size=6mm,
      inner sep=0.5pt,
      ball color=red!50!blue,
      text=white,},
    % Define style for FlyZhyRectangle
    FlyZhyRectangle/.style={
      rectangle,
      rounded corners,
      minimum size=6mm,
      ball color=red!50!blue,
      text=white,},
    % Define style for zhyfly
    zhyfly/.style={
      rectangle,
      rounded corners,
      minimum size=6mm,
      ball color=red!25!blue,
      text=white,},
    % Define style for new rectangle
    nrectangle/.style={
      rectangle,
      draw=#1!50,
      fill=#1!20,
      minimum size=5mm,
      inner sep=0.1pt,}
}
\ctikzset{
  bipoles/length=.8cm
}
% code
\lstnewenvironment{VHDLcode}[1][]{%
  \lstset{
    basicstyle=\footnotesize\ttfamily\color{black},%
    columns=flexible,%
    framexleftmargin=.7mm,frame=shadowbox,%
    rulesepcolor=\color{blue},%
%    frame=single,%
    backgroundcolor=\color{yellow!20},%
    xleftmargin=1.2\fboxsep,%
    xrightmargin=.7\fboxsep,%
    numbers=left,numberstyle=\tiny\color{blue},%
    numberblanklines=false,numbersep=7pt,%
    language=VHDL%
    }\lstset{#1}}{}
\lstnewenvironment{VHDLmiddle}[1][]{%
  \lstset{
    basicstyle=\scriptsize\ttfamily\color{black},%
    columns=flexible,%
    framexleftmargin=.7mm,frame=shadowbox,%
    rulesepcolor=\color{blue},%
%    frame=single,%
    backgroundcolor=\color{yellow!20},%
    xleftmargin=1.2\fboxsep,%
    xrightmargin=.7\fboxsep,%
    numbers=left,numberstyle=\tiny\color{blue},%
    numberblanklines=false,numbersep=7pt,%
    language=VHDL%
    }\lstset{#1}}{}
\lstnewenvironment{VHDLsmall}[1][]{%
  \lstset{
    basicstyle=\tiny\ttfamily\color{black},%
    columns=flexible,%
    framexleftmargin=.7mm,frame=shadowbox,%
    rulesepcolor=\color{blue},%
%    frame=single,%
    backgroundcolor=\color{yellow!20},%
    xleftmargin=1.2\fboxsep,%
    xrightmargin=.7\fboxsep,%
    numbers=left,numberstyle=\tiny\color{blue},%
    numberblanklines=false,numbersep=7pt,%
    language=VHDL%
    }\lstset{#1}}{}
% pdf
\hypersetup{pdfpagemode=FullScreen,%
            pdfauthor={Haiyong Zheng},%
            pdftitle={Title},%
            CJKbookmarks=true,%
            bookmarksnumbered=true,%
            bookmarksopen=false,%
            plainpages=false,%
            colorlinks=true,%
            citecolor=green,%
            filecolor=magenta,%
            linkcolor=cyan,%red(default)
            urlcolor=cyan}
% section
%http://tex.stackexchange.com/questions/34288/how-to-place-a-shaded-box-around-a-section-label-and-name
\newcommand\titlebar{%
\tikz[baseline,trim left=3.1cm,trim right=3cm] {
    \fill [cyan!25] (2.5cm,-1ex) rectangle (\textwidth+3.1cm,2.5ex);
    \node [
        fill=cyan!60!white,
        anchor= base east,
        rounded rectangle,
        minimum height=3.5ex] at (3cm,0) {
        \textbf{\thesection.}
    };
}%
}
\titleformat{\section}{\Large\bf\color{blue}}{\titlebar}{0.1cm}{}
% head/foot
\setlength{\headheight}{15pt}
\pagestyle{fancy}
\fancyhf{}
%\lhead{\color{black!50!green}2014年秋季学期}
\chead{\color{black!50!green}Zooplankton Attributes}
%\rhead{\color{black!50!green}通信电子电路}
\lfoot{\color{blue!50!green}王如晨\ 朱亚菲}
\cfoot{\color{blue!50!green}\href{http://vision.ouc.edu.cn/~zhenghaiyong}{CVBIOUC}}
\rfoot{\color{blue!50!green}$\cdot$\ \thepage\ $\cdot$}
\renewcommand{\headrulewidth}{0.4pt}
\renewcommand{\footrulewidth}{0.4pt}


%%---------------------------------------------------------------------
\begin{document}
%%---------------------------------------------------------------------
%%---------------------------------------------------------------------
% \titlepage
\title{\vspace{-2em}浮游动物特征总结\vspace{-0.7em}}
\author{王如晨\ 朱亚菲}
\date{\vspace{-0.7em}2015年8月\vspace{-0.7em}}
%%---------------------------------------------------------------------
\maketitle\thispagestyle{fancy}
%%---------------------------------------------------------------------
\maketitle
\tableofcontents 

形态学是生物分类学的主要依据。

\section{PkID中用到的特征}
\section{计算机视觉特征提取}

\subsection{几何参数}

\subsubsection{边界的周长}
轮廓边界的周长。对轮廓边缘上的像素点的统计。

\subsubsection{边界的曲率}

\subsubsection{面积}
描述区域大小的特征。对区域内总像素点的统计。

\subsubsection{宽度和高度}
最小外接矩形的宽度和高度

\subsubsection{矩形度}
反映被检测目标的最小外接矩形的充满程度,当目标的形状越接近矩形时,矩形度的值越接近1。
    \begin{displaymath}
    R=\frac{A}{WH}
    \end{displaymath}
    A为目标的面积,W、H分别为最小外接矩形的宽度和高度。

\subsubsection{体态比}
为目标最小外接矩形的长与宽的比值。
    \begin{displaymath}
    C=\frac{W}{H}
    \end{displaymath}
    
\subsubsection{圆形性}
用目标区域的所有边界点定义的特征向量。
    \begin{displaymath}
    C_{I}=\frac{\mu_{R}}{\sigma_{R}}
    \end{displaymath}
    $\mu_{R}$为区域重心到边界点的平均距离,$\sigma_{R}$为从区域重心到边界点的距离的平均方差。

\subsubsection{偏心率}
在一定程度上反映了区域的紧凑程度。定义为目标区域长短主轴的平方根的比值。
    \begin{displaymath}
    E=\frac{p}{q}
    \end{displaymath}
    设目标区域在XY平面上,区域像素点绕X轴的转动惯量为A,绕Y轴的转动惯量为B,惯性积为C。目标区域的长度分别是p和q。
    \begin{displaymath}
    p=\sqrt{\frac{2}{(A+B)+\sqrt{(A-B)^{2}+4C^{2}}}}
    \end{displaymath}
    \begin{displaymath}
    q=\sqrt{\frac{2}{(A+B)-\sqrt{(A-B)^{2}+4C^{2}}}}
    \end{displaymath}
    
\subsubsection{凸率}
为目标区域面积与目标区域凸包面积之比,该特征包含着描述边界不规则特性的信息。
    \begin{displaymath}
    C_{R}=\frac{A}{\sum_{x=1}^{M}\sum_{y=1}^{N}k(x,y)}
    \end{displaymath}
    分母为凸包区域的面积。
    
\subsubsection{密集度}
描述目标密集度的量化特征,提供了目标形状的重要信息。在周长确定后,密集度越高,所围成的面积越大。
    \begin{displaymath}
    C_{2}=\frac{L^{2}}{4\pi A}
    \end{displaymath}
    L为周长。
    
\subsubsection{球状性}
内切圆的直径与外接圆的直径之比。
    \begin{displaymath}
    S=\frac{r_{i}}{r_{c}}
    \end{displaymath}
    
\subsubsection{伸长度}
周长与目标区域最小外接矩形面积之比。
    \begin{displaymath}
    P=\frac{L}{WH}
    \end{displaymath}

\subsubsection{叶状性}
叶状反映了边界的幅度特征,为区域重心到边界的最短距离与目标区域的最大宽度之比。
    \begin{displaymath}
    B=\frac{R_{1}}{W_{max}}
    \end{displaymath}
    
\subsection{几种典型的特征描述方法}

\subsubsection{边界描述子}
\begin{itemize}
\item 链码
\item 多边形近似
\item 骨架
\item 形状数
\item 统计矩:边界线段的形状可以通过简单的统计矩进行定量的描述,如均值、方差和高阶矩。
\item 傅里叶描述子
\item 曲率尺度空间
\item 形状上下文(KNN)
\end{itemize}

\subsubsection{区域描述子}
\begin{itemize}
\item 拓扑描述:欧拉数
\item 不变矩
\item 角半径变换(Angular RadialTransformation, ART):通过使用一组半径变换系数,描述单个连通区域或者不连通区域,对旋转和噪声具有鲁棒性。
\item 纹理
    \begin{itemize}
    \item 统计方法:灰度共生矩阵
    \item 模型法:马尔科夫随机场
    \item 频谱方法:Gabor滤波、小波变换
    \end{itemize}
\end{itemize}

\begin{comment}
\subsection{经典特征描述方法}

\subsubsection{SIFT特征}

\subsubsection{HOG特征}

\subsubsection{LBP特征}

\subsubsection{Shape Context}

\subsubsection{Fisher Vector}
\end{comment}

\subsection{特征融合}
特征融合分为三个层次:数据级融合、特征级融合和决策级融合。
数据级融合是结合未加工的信息来得到更加丰富的信息。
特征级融合是选择并结合特征来去除多余和无关的特征。
决策级融合是用多个相同或不同的分类器,相同或不同的分类器。

图像融合方法:

像素级:PCA(主成分分析)、HIS变换、Brovery变换、线性加权法、SFIM、IHS变换、高通滤波法、小波变换融合算法。

特征级:聚类分析法、贝叶斯估计法、信息熵法、神经网络法、带权平均法、Dempster-shafer推理法、表决法及神经网络法。

决策级:神经网络法、贝叶斯融合、模糊聚类法、模糊集理论、可靠性理论以及逻辑模板法。

\subsubsection{贝叶斯融合(Bayes Fusion)}
当在进行图像分类过程中,可能需要用到不止一种特征。贝叶斯融合可以通过结合不同分类器的结果实现特征融合。

\textbf{贝叶斯融合规则:}

如果要将一幅图像分类到$n$个可能的种类中($\omega_{1},\dots,\omega_{n}$),$x_{i}$表示第$i$个分类器产生的待识的属性,它属于$n$个模式类之一。记$P(\omega_{k})$为先验概率,$P(x_{i}|\omega_{k})$为每类的概率密度函数,$P(x_{1},\dots,x_{R}|\omega_{k})$联合概率分布函数,$R$为用来分类的分类器数目。

根据贝叶斯最小错误率理论,如果
\begin{equation}
P(\omega_{j}|x_{1},\dots,x_{R})=\max_{k}P(\omega_{k}|x_{1},\dots,x_{R})
\end{equation}
则$Z\in\omega_{j}$。

且有:
\begin{equation}
P(\omega_{j}|x_{1},\dots,x_{R})=\frac{P(x_{1},\dots,x_{R}|\omega_{k})P(\omega_{k})}{P(x_{1},\dots,x_{R})}
\end{equation}
其中
\begin{equation}
P(x_{1},\dots,x_{R})=\sum^{n}_{j=1}P(x_{1},\dots,x_{R}|\omega_{k})P(\omega_{j})
\end{equation}
假定分类器度量之间是相互独立的,有:
\begin{equation}
P(x_{1},\dots,x_{R}|\omega_{k})=\prod^{R}_{i=1}P(x_{i}|\omega_{k})
\end{equation}
将式(3)(4)带入式(2)有:
\begin{equation}
P(\omega_{j}|x_{1},\dots,x_{R})=\frac{P(\omega_{k})\prod^{R}_{i=1}P(x_{i}|\omega_{k})}{\sum^{n}_{j=1}P(\omega_{j})\prod^{R}_{i=1}P(x_{i}|\omega_{k})}
\end{equation}
将式(5)带入式(1),得到:
\begin{enumerate}
\item 融合规则1:乘法规则
    \begin{equation}
    P(\omega_{j})\prod^{R}_{i=1}P(x_{i}|\omega_{j})=\max^{n}_{k=1}P(\omega_{k})\prod^{R}_{i=1}P(x_{i}|\omega_{k})
    \end{equation}
    转化为后验概率,得到:
    \begin{equation}
    P^{-(R-1)}(\omega_{j})\prod^{R}_{i=1}P(\omega_{j}|x_{i})=\max^{n}_{k=1}P^{-(R-1)}(\omega_{k})\prod^{R}_{i=1}P(\omega_{k}|x_{i})
    \end{equation}
    这样就可以将该图片归类为$\omega_{j}$。
\item 融合规则2:加法规则

    在乘法规则中,如果假定由分类器输出的后验概率与相应的先验概率之间只有微小的偏差:
    \begin{equation}
    P(\omega_{k}|x_{i})=P(\omega_{k})(1+\delta_{ki}), \delta_{ki}\ll1
    \end{equation}
    由此得到:
    \begin{equation}
    P^{-(R-1)}(\omega_{k})\prod^{R}_{i=1}P(\omega_{k}|x_{i})=P(\omega_{k})\prod^{R}_{i=1}(1+\delta_{ki})
    \end{equation}
    如果将上式右边的乘积展开忽略二次以上的项,即:
    \begin{equation}
    P(\omega_{k})\prod^{R}_{i=1}(1+\delta_{ki})=P(\omega_{k})+P(\omega_{k})\sum^{R}_{i=1}\delta_{ki}
    \end{equation}
    再将式(8)(10)带入(7)可以得到加法规则如下:
    \begin{equation}
    (1-R)P(\omega_{j})+\sum^{n}_{i=1}P(\omega_{j}|x_{i})=\max^{n}_{k=1}[(1-R)P(\omega_{k})+\sum^{R}_{i=1}P(\omega_{k}|x_{i})]
    \end{equation}
    通常我们假定各类的先验概率是相等的,则上式等价于:
    \begin{equation}
    \sum^{R}_{i=1}P(\omega_{j}|x_{i})=\max^{n}_{k=1}\sum^{R}_{i=1}P(\omega_{k}|x_{i})
    \end{equation}
    由于加法规则对估计误差不敏感,因而具有比其他规则更好的性能。
\end{enumerate}


\begin{comment}
\begin{itemize}
\item 基于灰度共生矩阵的方法
\item 灰度-梯度共生矩阵分析法
\item 灰度行程长度统计法
\item 小波分析法
\item 基于Gabor小波变换的纹理分析法
\end{itemize}
\end{comment}



\bibliographystyle{plain}

\bibliography{Features} %参考文献

\end{document}
