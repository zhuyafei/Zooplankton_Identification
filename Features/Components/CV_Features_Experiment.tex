\section{选取特征进行实验}
该部分进行的实验:用Matlab和C(以MATLAB为主)实现对浮游动物特征的提取(特征包括PkID中部分特征以及计算视觉中的一些特征提取方法),并进行分类。在该实验中使用的去噪方法是去掉连通区域小于50的噪声。

%\begin{comment}
%\subsubsection{实验一}
%\begin{description}
%\item[去噪方法:] 去掉连通区域小于20的噪声。
%\item[选用特征:] Mean、StdDev、CV、SR、MeanPos、Elongation、Circ、Feret、PerimAreaexc、CDexc、Skelarea、FeretAreaexc、PerimFeret、矩形度、体态比、凸率、伸长度、灰度共生矩阵(对比度)、对称性(左右),前13个特征为PkID系统中使用的部分特征。
%\item[分类器:] 采用随机森林进行训练和分类得到的结果如图~\ref{fig:19-Features-MATLAB-our-RF},其分类准确率为71.9\%。采用SVM Linear进行训练和分类得到的结果如图~\ref{fig:19-Features-MATLAB-our-SVM-Linear},其分类准确率为58.4\%
%\end{description}
%\begin{figure}[!ht]
%\centering
%\includegraphics[width=1.0\linewidth]{19-Features-MATLAB-our20-RF}
%\caption{Matlab-19个特征采用随机森林进行分类的结果}
%\label{fig:19-Features-MATLAB-our20-RF}
%\end{figure}
%
%\begin{figure}[!ht]
%\centering
%\includegraphics[width=1.0\linewidth]{19-Features-MATLAB-our20-SVM-Linear}
%\caption{Matlab-19个特征采用SVM Linear进行分类的结果}
%\label{fig:19-Features-MATLAB-our20-SVM-Linear}
%\end{figure}
%\end{comment}
\subsection{参数特征选取实验}
\subsubsection{实验一}
\label{shiyan1-features}
\begin{description}
\item[选用特征:] Mean、StdDev、CV、SR、MeanPos、Elongation、Circ、Feret、PerimAreaexc、CDexc、Skelarea、FeretAreaexc、PerimFeret。(这些特征是从PkID的67个特征中选取的)
\item[分类器:] 随机森林、SVM
\end{description}
\begin{itemize}
\item MATLAB:采用随机森林进行训练和分类得到的结果如图~\ref{fig:13-Features-MATLAB-our-RF},其分类准确率为61.6\%。采用SVM Linear进行训练和分类得到的结果如图~\ref{fig:13-Features-MATLAB-our-SVM-Linear},其分类准确率为39.9\%
\item C:采用随机森林进行训练和分类得到的结果如图~\ref{fig:13-Features-C-our-RF},其分类准确率为59.7\%。采用SVM Linear进行训练和分类得到的结果如图~\ref{fig:13-Features-C-our-SVM-Linear},其分类准确率为33.4\%
\end{itemize}

\begin{figure}[!ht]
\centering
\includegraphics[width=1.0\linewidth]{13-Features-MATLAB-our-RF}
\caption{Matlab-13个特征采用随机森林进行分类的结果}
\label{fig:13-Features-MATLAB-our-RF}
\end{figure}

\begin{figure}[!ht]
\centering
\includegraphics[width=1.0\linewidth]{13-Features-MATLAB-our-SVM-Linear}
\caption{Matlab-13个特征采用SVM Linear进行分类的结果}
\label{fig:13-Features-MATLAB-our-SVM-Linear}
\end{figure}

\begin{figure}[!ht]
\centering
\includegraphics[width=1.0\linewidth]{13-Features-C-our-RF}
\caption{C-13个特征采用随机森林进行分类的结果}
\label{fig:13-Features-C-our-RF}
\end{figure}

\begin{figure}[!ht]
\centering
\includegraphics[width=1.0\linewidth]{13-Features-C-our-SVM-Linear}
\caption{C-13个特征采用SVM Linear进行分类的结果}
\label{fig:13-Features-C-our-SVM-Linear}
\end{figure}
%根据实验一、二的分析:去掉小连通区域的噪声可以提高分类准确率,但是去掉的小连通区域面积阈值设置相差不大时不会对实验结果产生很大影响。

\subsubsection{实验二}
\label{shiyan2-features}
\begin{description}
\item[选用特征:] Mean、StdDev、CV、SR、MeanPos、Elongation、Circ、Feret、PerimAreaexc、CDexc、Skelarea、FeretAreaexc、PerimFeret、矩形度、体态比、凸率、伸长度、灰度共生矩阵(对比度)、对称性(左右),共19个特征。(前13个特征为\ref{shiyan1-features}实验一所使用的特征)
\item[分类器:] 随机森林、SVM

MATLAB:采用随机森林进行训练和分类得到的结果如图~\ref{fig:19-Features-MATLAB-our-RF},其分类准确率为72.9\%。采用SVM Linear进行训练和分类得到的结果如图~\ref{fig:19-Features-MATLAB-our-SVM-Linear},其分类准确率为58.9\%
\end{description}
\begin{figure}[!ht]
\centering
\includegraphics[width=1.0\linewidth]{19-Features-MATLAB-our-RF}
\caption{Matlab-19个特征采用随机森林进行分类的结果}
\label{fig:19-Features-MATLAB-our-RF}
\end{figure}

\begin{figure}[!ht]
\centering
\includegraphics[width=1.0\linewidth]{19-Features-MATLAB-our-SVM-Linear}
\caption{Matlab-19个特征采用SVM Linear进行分类的结果}
\label{fig:19-Features-MATLAB-our-SVM-Linear}
\end{figure}
%根据实验一、二的分析:去掉小连通区域的噪声可以提高分类准确率,但是去掉的小连通区域面积阈值设置相差不大时不会对实验结果产生很大影响。

\subsubsection{实验三}
\label{shiyan3-features}
\begin{description}
\item[选用特征:] 在\ref{shiyan2-features}实验二所用特征的基础上增加的了不变矩特征,共20个特征。
\item[分类器:] 随机森林、SVM

MATLAB:采用随机森林进行训练和分类得到的结果如图~\ref{fig:20-Features-MATLAB-our-RF},其分类准确率为73.7\%。采用SVM Linear进行训练和分类得到的结果如图~\ref{fig:20-Features-MATLAB-our-SVM-Linear},其分类准确率为61.0\%
\end{description}
\begin{figure}[!ht]
\centering
\includegraphics[width=1.0\linewidth]{20-Features-MATLAB-our-RF}
\caption{Matlab-20个特征采用随机森林进行分类的结果}
\label{fig:20-Features-MATLAB-our-RF}
\end{figure}

\begin{figure}[!ht]
\centering
\includegraphics[width=1.0\linewidth]{20-Features-MATLAB-our-SVM-Linear}
\caption{Matlab-20个特征采用SVM Linear进行分类的结果}
\label{fig:20-Features-MATLAB-our-SVM-Linear}
\end{figure}

\subsubsection{实验四}
\label{shiyan4}
\begin{description}
\item[选用特征:] 采用内距离形状上下文提取图像特征:
    \begin{enumerate}
    \item 先从图像中挑选39张图像作为模板(每种浮游动物中选取3张)。
    \item 采用内距离形状上下文分别计算训练集和测试集中所有图像和上一步中39张图像间的距离。
    \item 将上面计算得到的距离矩阵作为训练集和测试集的特征,输入到ELM中进行学习和分类。
    \end{enumerate}
\item[分类器:] ELM(经过实验发现ELM的分类效果要比SVM和随机森林的好)

MATLAB:采用ELM进行训练和分类得到的结果如图~\ref{fig:IDSC-39-Features-MATLAB-ELM},其分类准确率为63.4\%。
\end{description}
\begin{figure}[!ht]
\centering
\includegraphics[width=1.0\linewidth]{IDSC-39-Features-MATLAB-ELM}
\caption{Matlab-IDSC特征采用ELM进行分类的结果(39张图像作为模板)}
\label{fig:IDSC-39-Features-MATLAB-ELM}
\end{figure}

\subsubsection{实验五}
\begin{description}
\item[选用特征:] 采用内距离形状上下文提取图像特征。该实验在\ref{shiyan4}实验四的基础上,将39张图像模板增加到65张。
\item[分类器:] ELM

MATLAB:采用ELM进行训练和分类得到的结果如图~\ref{fig:IDSC-65-Features-MATLAB-ELM},其分类准确率为64.8\%。
\end{description}
\begin{figure}[!ht]
\centering
\includegraphics[width=1.0\linewidth]{IDSC-65-Features-MATLAB-ELM}
\caption{Matlab-IDSC特征采用ELM进行分类的结果(65张图像作为模板)}
\label{fig:IDSC-65-Features-MATLAB-ELM}
\end{figure}

\begin{figure}[!ht]
\centering
\includegraphics[width=1.0\linewidth]{20-Features-MATLAB-our-SVM-Linear}
\caption{Matlab-20个特征采用SVM Linear进行分类的结果}
\label{fig:20-Features-MATLAB-our-SVM-Linear}
\end{figure}


\subsection{特征融合方法实验}
该部分进行的是特征融合实验,确定哪种特征融合方式可以得到更高的分类准确率。

\subsubsection{实验一(特征融合方法一)}
\label{ronghe1}
该实验进行的是特征融合。由于在实验一——三中使用的特征都是特征值,而实验四中使用的特征以及HOG、LBP和其他的一些特征都是特征向量的形式。如果要将这些特征一起使用就需要进行特征融合。在该实验中,将\ref{shiyan3-features}中的20个特征和LBP特征融合,具体的融合方法:
\begin{enumerate}
\item 用训练集不同种的特征(这里的将特征分为两种:实验三中20个特征作为一种,LBP特征作为一种)分别进行训练得到分类器(20个特征采用随机森林进行训练,LBP采用SVM进行训练)。然后将训练集对应的这两种特征分别输入到其对应的分类器中进行预测,这两种特征会分别得到训练集中每个样本属于每个类别的分类概率($m \times n$维,m为训练集样本数,n为类别数)。
\item 将每种特征得到的概率进行拼接($m \times 2n$维),再输入到分类器(这里的分类器使用的是SVM)进行训练。
\item 然后将测试集的分类概率(用和步骤1中同样的方法可以得到测试集的分类概率)输入到步骤2中训练好的分类器,得到最终的分类结果。
\end{enumerate}

在该实验得到的分类结果如图\ref{fig:20+LBP-fusion1-matlab},其分类准确率为76.1\%。
\begin{figure}[!ht]
\centering
\includegraphics[width=1.0\linewidth]{20+LBP-fusion1-matlab}
\caption{Matlab-20个特征和LBP特征融合方法一}
\label{fig:20+LBP-fusion1-matlab}
\end{figure}

\subsubsection{实验二(特征融合方法二)}
该实验也是将\ref{shiyan3-features}中的20个特征和LBP特征融合,采用的融合方法:
\begin{enumerate}
\item 用训练集不同种的特征(这里的将特征分为两种:实验三中的20个特征作为一种,LBP特征作为一种)分别进行训练得到分类器(20个特征采用随机森林进行训练,LBP采用SVM进行训练)。
\item 计算每个分类器的权重:对于训练样本集中的每一个样本,分别将其每种特征输入到对应的特征分类器中进行识别,如果能够识别正确,则其对应的特征分类器的权重加一,最终得到每种特征的权重。
\item 预测概率:每幅的不同种类特征通过分类器可以得到其属于每个类别的分类概率(与特征融合方法一中步骤1相同)。根据权重和分类概率,计算出最终属于各个类别的概率。
\end{enumerate}

在该实验得到的分类结果如图\ref{fig:20+LBP-fusion2-matlab},其分类准确率为73.7\%。
\begin{figure}[!ht]
\centering
\includegraphics[width=1.0\linewidth]{20+LBP-fusion2-matlab}
\caption{Matlab-20个特征和LBP特征融合方法二}
\label{fig:20+LBP-fusion2-matlab}
\end{figure}

\subsection{融合不同特征实验}
该部分实验采用特征融合方法一。在实验过程中,采用的针对不同特征采用的分类器:
\begin{itemize}
\item \ref{shiyan3-features}实验三中的20个特征:随机森林
\item LBP特征、Gabor特征、Fourier描述子、SIFT特征:SVM
\item IDSC:ELM
\end{itemize}
\subsubsection{实验一}
\begin{description}
\item[选用特征:] 在\ref{ronghe1}特征融合方法一的基础上,将\ref{shiyan3-features}中的20个特征、LBP特征融和Gabor特征融合。
\end{description}
MATLAB:在该实验得到的分类结果如图\ref{fig:20+LBP+Gabor-Features-MATLAB},其分类准确率为73.6\%。
\begin{figure}[!ht]
\centering
\includegraphics[width=1.0\linewidth]{20+LBP+Gabor-Features-MATLAB}
\caption{Matlab-20个特征、LBP和Gabor特征融合方法一}
\label{fig:20+LBP+Gabor-Features-MATLAB}
\end{figure}

\subsubsection{实验二}
\begin{description}
\item[选用特征:] 在\ref{ronghe1}特征融合方法一的基础上,将\ref{shiyan3-features}中的20个特征、LBP特征融Fourier描述子融合。
\end{description}
MATLAB:在该实验得到的分类结果如图\ref{fig:20+LBP+Fourier-Features-MATLAB},其分类准确率为76.2\%。
\begin{figure}[!ht]
\centering
\includegraphics[width=1.0\linewidth]{20+LBP+Fourier-Features-MATLAB}
\caption{Matlab-20个特征、LBP和Fourier描述子融合方法一}
\label{fig:20+LBP+Fourier-Features-MATLAB}
\end{figure}

\subsubsection{实验三}
\begin{description}
\item[选用特征:] 在\ref{ronghe1}特征融合方法一的基础上,将\ref{shiyan3-features}中的20个特征、LBP特征和SIFT特征融合。
\end{description}
MATLAB:在该实验得到的分类结果如图\ref{fig:20+LBP+SIFT-Features-MATLAB},其分类准确率为76.1\%。
\begin{figure}[!ht]
\centering
\includegraphics[width=1.0\linewidth]{20+LBP+SIFT-Features-MATLAB}
\caption{Matlab-20个特征、LBP和SIFT特征融合方法一}
\label{fig:20+LBP+SIFT-Features-MATLAB}
\end{figure}

\subsubsection{实验四}
\begin{description}
\item[选用特征:] 在\ref{ronghe1}特征融合方法一的基础上,将\ref{shiyan3-features}中的20个特征和\ref{shiyan4}中的IDSC特征融合。
\end{description}
MATLAB:在该实验得到的分类结果如图\ref{fig:20+IDSC-Features-MATLAB},其分类准确率为76.3\%。
\begin{figure}[!ht]
\centering
\includegraphics[width=1.0\linewidth]{20+IDSC-Features-MATLAB}
\caption{Matlab-20个特征和IDSC特征融合方法一}
\label{fig:20+IDSC-Features-MATLAB}
\end{figure}

\subsubsection{实验五}
\begin{description}
\item[选用特征:] 在\ref{ronghe1}特征融合方法一的基础上,将\ref{shiyan3-features}中的20个特征、LBP特征和\ref{shiyan4}中的IDSC特征融合。(采用特征融合方法一)
\end{description}
MATLAB:在该实验中,IDSC采用39个模板得到的分类结果如图\ref{fig:20+IDSC39+LBP-Features-MATLAB},其分类准确率为77.1\%。
\begin{figure}[!ht]
\centering
\includegraphics[width=1.0\linewidth]{20+IDSC39+LBP-Features-MATLAB}
\caption{Matlab-20个特征、LBP和IDSC特征融合方法一}
\label{fig:20+IDSC39+LBP-Features-MATLAB}
\end{figure}

MATLAB:在该实验中,IDSC采用65个模板得到的分类结果如图\ref{fig:20+IDSC65+LBP-Features-MATLAB},其分类准确率为77.7\%。
\begin{figure}[!ht]
\centering
\includegraphics[width=1.0\linewidth]{20+IDSC65+LBP-Features-MATLAB}
\caption{Matlab-20个特征、LBP和IDSC特征融合方法一}
\label{fig:20+IDSC65+LBP-Features-MATLAB}
\end{figure}

\subsection{ELM作为分类器的实验}
该部分使用极限学习机(ELM)作为分类器。

\subsubsection{实验一}
\begin{description}
\item[选用特征:] 采用\ref{shiyan3-features}中的20个特征进行实验。
\end{description}
MATLAB:在该实验中ELM隐藏神经元设置为650个,实验得到的分类结果如图\ref{fig:20-Features-ELM-MATLAB},其分类准确率为72.4\%。
\begin{figure}[!ht]
\centering
\includegraphics[width=1.0\linewidth]{20-Features-ELM-MATLAB}
\caption{Matlab-20个特征采用ELM进行分类}
\label{fig:20-Features-ELM-MATLAB}
\end{figure}


\section{实验总结}
实验分析:
\begin{itemize}
\item 实验发现,\ref{shiyan3-features}中20个特征、LBP特征和IDSC特征对分类准确率提高有较大作用,其中最好的是\ref{shiyan3-features}中20个特征。
\item IDSC通常采用K-NN进行分类,但是计算时间较长,因此在实验中没有采用K-NN进行分类。而是通过选取模板,计算所有图像和模板之间的距离,作为图像的特征。
\item 采用IDSC提取图像特征时,先选用了39张图像作为模板,后来又选用65张图像作为模板。模板图像数量越多,分类的准确率越高,但同时需要的时间越多。
\end{itemize}
接下来的工作:
\begin{itemize}
\item 尝试在采用IDSC提取图像特征时,继续增加模板图像的数量。
\item 目前采用的融合方法在融合过程中,这三种特征所占的权重是一样的。下一步想更改特征融合方法,根据这三种特征对提高分类准确率的贡献,在融合过程中设置不同的权重。
\item 由于目标物体轮廓不平滑,骨架提取的结果比较差。想要寻找可以平滑物体轮廓边缘的方法,方便提取较好的骨架。
\item 不同的特征适用的分类器不同,分类器也会影响分类准确率的提高。目前使用的分类器有SVM、随机森林和ELM,想要寻找更加合适的分类器。
\item 寻找特征描述方法。
\end{itemize}

\begin{comment}
2015.9.15
\begin{itemize}
\item MATLAB进行实验的结果要比C的好,因此实验以MATLAB为主。
\item 在进行实验过程中,PKID中的特征只使用了13个(原来计划选用22个)。原因是由于图像质量原因,特征提取效果不是特别好,以至于一些特征对分类准确率的提升没有太大贡献。另外,还有两个特征我没有找到计算方法,但我感觉对实验结果提高不会有太大影响。
\item 实验中提取的特征不如PkID系统中的特征好(有一部分原因是实验中特征提取是对图像提取的,而图像质量又较低,导致提取的特征不准确。PkID系统中的基本特征都是从ZooProcess中得到的,能更好的反映浮游动物的特征),其分类的准确率也比PkID系统中的特征低2到3个百分点。
\item 因为需要使用的特征既有特征值,也有特征向量,这些特征直接一起使用会降低准确率,因此需要进行了特征融合(实验四、五)。目前采用了20个特征与LBP特征融合,分类的准确率有了一定的提高(达到76.1\%)。
\end{itemize}

接下来的工作:在特征融合的基础上继续进行实验,实验将不同形式的特征结合来得到更高的分类准确率。
\end{comment}